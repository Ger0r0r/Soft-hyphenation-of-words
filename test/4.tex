\documentclass[a4paper,12pt]{article}	% Стиль
\usepackage[pdftex,unicode, 
			colorlinks=true,
			linkcolor = blue]{hyperref}	% нумерование страниц, ссылки!!!!ИМЕННО В ТАКОМ ПОРЯДКЕ СО СЛЕДУЮЩИМ ПАКЕТОМ
\usepackage[warn]{mathtext}				% Поддержка русского текста в формулах
\usepackage[T1, T2A]{fontenc}			% Пакет выбора кодировки и шрифтов
\usepackage[utf8]{inputenc} 			% любая желаемая кодировка
\usepackage[russian,english]{babel}		% поддержка русского языка
\usepackage{wrapfig}					% Плавающие картинки
\usepackage{amssymb, amsmath}			% Плавающие картинки
\usepackage{graphicx}

\ifpdf
        \usepackage{cmap} 				% чтобы работал поиск по PDF
        \usepackage[pdftex]{graphicx}
        \usepackage{pgfplotstable}		% Для вставки таблиц.
        \pdfcompresslevel=9 			% сжимать PDF
\else
        \usepackage{graphicx}
\fi

\graphicspath {{ Images / }}

\usepackage[left=2cm,right=2cm,top=2cm,bottom=2cm]{geometry}


\begin{document} 
\textbf{4.4.4}

\textbf{Условие:}
Определите установившуюся скорость движения шайбы массы $m$ и радиуса $R$ по наклонной плоскости, образующей угол $\alpha$ с горизонтом, в случае, когда между шайбой и плоскостью имеется слой смазки толщиной $\Delta$ и вязкости $\eta$.

\textbf{Решение:}
\begin{figure}[htp]
	\centering
	\includegraphics[width=7cm]{4_4_4-1.png}
\end{figure}

В данной задаче считаем, что шайба не тонет в слое смазке и что на шайбу не дей\-ству\-ет сила поверхностного натяжения.

Слой смазки вблизи шайбы можно считать "прилипшим" к шайбе. На шайбу бу\-дет действовать сила сопротивления вязкого трения, которая зависит от скорости. Тем са\-мым, с течением времени скорость шайбы установится и будет постоянна. В этом случае, от дна шайбы до поверхности под шайбой будет виден градиент скорости: у шайбы -- ско\-рость шайбы; у поверхности -- скорость нулевая. Тем самым на шайбу будет действовать сила сопротивления вязкого трения, которая возникает в смазке.

По II закон Ньютона на ось параллельную плоскости получаем уравнение:
$$F_{сп}=mg\sin\alpha$$

С другой стороны, эта сила выражается через параметры смазки и скорость шайбы.
$$F_{сп}=\eta\frac{\upsilon S}{\Delta}$$
$$mg\sin\alpha=\eta\frac{\upsilon S}{\Delta}$$
$$\upsilon=\frac{mg\Delta\sin\alpha}{\eta S}$$
$$\boxed{\upsilon=\frac{mg\Delta\sin\alpha}{\pi R^2\eta}}$$

\large!\footnotesize Для достоверности, стоит отметить, что мой ответ отличается от ответа, который дан автором в задачнике. В ответ на эти претензии, я хочу заметить, что этот ответ содержит не сообщённую в условии величину $h$, потому считаю этот ответ не состоятельным. Если вы нашли верный путь к решению данной задачи и(или) вам известен секрет загадочного $h$, прошу связаться со мной.
\end{document}
